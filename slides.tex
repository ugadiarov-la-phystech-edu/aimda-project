\documentclass[aspectratio=169,8pt]{beamer}

\usepackage[T1,T2A]{fontenc}
\usepackage[utf8]{inputenc}
\usepackage[main=russian,english]{babel}
\usepackage[normalem]{ulem}
\usepackage{hyperref}
\usepackage{amsmath,amsthm,amssymb,lmodern}

\usetheme{Boadilla}
\setbeamertemplate{caption}{\insertcaption}

\title[Проект по дисциплине МИИАД] {Проект по дисциплине "Методы искусственного интеллекта в анализе данных"}
\subtitle{Этап 1}

\author[Бобровских, Иванов, Угадяров] {Бобровских Глеб, Иванов Дмитрий, Угадяров Леонид \\ \tiny\url{https://github.com/ugadiarov-la-phystech-edu/aimda-project}}
\institute{Группа 4}


\begin{document}

\begin{frame}
\titlepage
\end{frame}

\begin{frame}
\frametitle{Набор данных и применённые модели}

Рассматривается подвыборка за 2014 год из набора данных об убийствах в США \href{https://www.kaggle.com/murderaccountability/homicide-reports}{\color{blue}{Homicide Reports, 1980-2014}}.
\newline Задача классификации --- предсказание значения бинарного признака \emph {Crime Solved}.
\newline Метрика качества --- \emph {F1} мера для класса нераскрытых преступлений.

\begin{block}{Метод опорных векторов}
В качестве реализации алгоритма использовался класс \emph {sklearn.svm.SVC} библиотеки \emph {Scikit-learn}. Наилучшие гиперпараметры\footnotemark:
\newline \emph {C = 0.1}, \emph {class\_weight = None}, \emph {coef0 = 0}, \emph {gamma = 'scale'}, \emph {kernel = 'linear'}
\end{block}

\begin{block}{Случайный лес}
В качестве реализации алгоритма использовался класс \emph {sklearn.ensemble.RandomForestClassifier} библиотеки \emph {Scikit-learn}. Наилучшие гиперпараметры\footnotemark[1]:
\newline \emph {criterion = 'gini'}, \emph {max\_depth = 10}, \emph {max\_features = 'auto'}, \emph {n\_estimators = 500}
\end{block}

\begin{block}{Бустинг}
В качестве реализации алгоритма использовалась класс \emph {catboost.CatBoostClassifier} библиотеки \emph {CatBoost}. Наилучшие гиперпараметры\footnotemark[1]:
\newline \emph {iterations = 300}, \emph {depth = 6}, \emph {loss\_function = 'Logloss'}, \emph {learning\_rate = 0.1}, \emph {l2\_leaf\_reg = 4.5}
\end{block}

\footnotetext{Значения остальных гиперпараметров оставлены по умолчанию}

\end{frame}

\begin{frame}
\frametitle{Результаты экспериментов}

Эксперименты проводились на платформе \emph{Google Colaboratory}. Характеристики предоставляемого оборудования:
\begin{itemize}
\item 2 ядра процессора Intel Xeon E5-2699 v4 2.20 ГГц
\item 12ГБ оперативной памяти
\end{itemize}

{\fontsize{7.5}{10}\selectfont {
\begin{table}
\centering
\caption{Метрики качества классификации обученных моделей}
\vspace{-5pt}
\begin{tabular}{|c|c|c|c|}
\hline
\  & \emph {F1} & \emph {Precision} & \emph {Recall} \\
\hline
\emph {SVC} & 0.735 $\pm$ 0.007 & 0.703 $\pm$ 0.005 & 0.770 $\pm$ 0.017\\
\hline
\emph {RandomForest} & 0.737 $\pm$ 0.009 & 0.725 $\pm$ 0.005 & 0.75 $\pm$ 0.02\\
\hline
\emph {CatBoost} & 0.748 $\pm$ 0.010 & 0.730 $\pm$ 0.008 & 0.767 $\pm$ 0.016\\
\hline
\end{tabular}
\end{table}
}}

{\fontsize{7.5}{10}\selectfont {
\begin{table}
\centering
\caption{Быстродействие обученных моделей}
\vspace{-5pt}
\begin{tabular}{|c|c|c|c|}
\hline
\  & \  & \  & \ \\
\  & \text{Время обучения, с} & $\dfrac{\text{Время обучения}}{\text{Количество объектов}}, \text{мс}$ \newline & \text{Время предсказания на одном объекте, мс} \\
\  & \  & \  & \ \\
\hline
\emph {SVC} & 16.4 & 1.89 & 1.3\\
\hline
\emph {RandomForest} & 3.4 & 0.39 & 37.2\\
\hline
\emph {CatBoost} & 4.7 & 0.54 & 1.5 \\
\hline
\end{tabular}
\end{table}
}}

\end{frame}


\end{document}
